\begin{abstract}
Existen diferentes tipos de estructuras de datos que permiten almacenar y actualizar una lista de elementos de diversas maneras, cada una teniendo sus ventajas y desventajas. La elección de una de estas estructuras puede ser un trabajo difícil de realizar pues esta a criterio del programador priorizar la complejidad temporal, el dinamismo del espacio o la facilidad en la comprensión del código, además de que la eficiencia de la estructura depende del problema a solucionar. Con el fin de facilitar el aprendizaje de como tomar esta decisión, realizamos un experimento en el cual solucionamos un problema usando dos tipos de estructuras de datos distintas y comparamos sus resultados, para posteriormente determinar cual de ambas es mas eficiente y por ende la estructura que se debe tomar para el programa.
\end{abstract}

% Note that keywords are not normally used for peerreview papers.
\begin{IEEEkeywords}
Estructura de datos, eficiencia algoritmica, complejidad temporal, complejidad espacial, programación.
\end{IEEEkeywords}

\IEEEpeerreviewmaketitle