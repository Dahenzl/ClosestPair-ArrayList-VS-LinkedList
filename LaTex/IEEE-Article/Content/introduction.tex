\section{Introducción}
\IEEEPARstart{L}{as} estructuras de datos son formas de organizar de manera eficiente una lista de elementos que el usuario necesite y se utilizan para almacenarlos de tal forma que se pueda acceder a ellos y actualizarlos de manera sencilla. Además, se utilizan para realizar diversos procesos con los elementos eficientemente, teniendo al alcance todos los que hacen parte de la estructura. Las estructuras de datos están divididas en estructuras estáticas, como las \textit{Array Lists}, estructuras dinámicas, como las pilas, las colas y las \textit{Linked Lists}, y estructuras no lineales como los árboles y los grafos. Dos de las estructuras mas comunes son las \textit{Array Lists} y las \textit{Linked Lists}. \cite{DataStructures}\\

Las \textit{Array Lists} almacenan elementos en ubicaciones de memoria contiguas, lo que causa que las direcciones de estos elementos sean fácilmente calculadas y esto permite un acceso más rápido a un elemento con un índice especifico, sin embargo, dado que los datos solo se pueden almacenar en bloques contiguos de memoria en una matriz, el tamaño de esta lista no se puede modificar en tiempo de ejecución, pues se corre el riesgo de sobrescribir otros datos. La asignación de memoria ocurre en tiempo de compilación. \cite{LinkedListVSArray}\\

Las \textit{Linked Lists} son estructuras de almacenamiento dinámicas y los elementos generalmente no se almacenan en ubicaciones contiguas, por lo que deben ser almacenados con un apuntador que haga referencia al siguiente elemento, por esta razón es que es necesario recorrer todos los elementos previos de la lista hasta tener acceso a un elemento en específico, sin embargo, dado que cada elemento apunta al siguiente, los datos pueden existir en direcciones dispersas, lo que permite un tamaño dinámico que puede cambiar durante la ejecución del programa, es decir que la asignación de memoria ocurre en tiempo de ejecución. \cite{parlante2001linked}\\

En este experimento vamos a determinar cual de estos dos tipos de estructura de datos es mas eficiente para resolver el problema de encontrar el par de puntos mas cercanos en una lista.
