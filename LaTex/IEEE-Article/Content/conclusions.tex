\section{Conclusiones}
Con la realización de este experimento pudimos determinar la importancia de considerar la implementación de diferentes tipos de estructuras de datos para nuestros programas, con el fin de escoger el que tenga mejor complejidad espacial y temporal, es decir que tenga una mayor eficiencia algorítmica, pues notamos en los resultados del experimento que, a pesar de que cuando se usa una misma técnica para el diseño de algoritmos el número  de iteraciones es similar sin importar el tipo de estructura que se haya utilizado, el tiempo de ejecución si varía mucho entre ellas, lo que nos permite elegir el tipo de estructura que mejor nos convenga.

Además, pudimos observar que, cuando el número de elementos crece considerablemente, los programas que usan \textit{Array Lists} son más eficientes temporalmente que los programas que usan \textit{Linked Lists}, pues notamos que las \textit{Array Lists} pueden acceder a un elemento especifico de la lista directamente con su índice, mientras que las \textit{Linked Lists} necesitan recorrer todos los elementos previos para acceder al dato que se necesita. Esto lo pudimos comprobar en las graficas realizadas a partir de los resultados del experimento, pues notamos que los métodos que usaron \textit{Array Lists} tuvieron una complejidad temporal de $O(N)$ y $O(N^2)$, mientras que los métodos que usaron \textit{Linked Lists} tuvieron una complejidad temporal de $O(N^2)$ y $O(N^3)$.

Por último, a partir de los resultados del experimento podemos concluir que los programas que usan la estructura estatica \textit{Array List} tienen una menor complejidad temporal, y por ende una mayor eficiencia algorítmica en la solución del problema de encontrar el par de puntos mas cercanos en una lista, que los programas que usan la estructura dinamica \textit{Linked List}.
